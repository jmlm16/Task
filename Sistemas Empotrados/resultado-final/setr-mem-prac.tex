\documentclass[a4paper, 11pt]{article}
\usepackage[utf8]{inputenc}
\usepackage[spanish]{babel}
\usepackage[margin=2.5cm]{geometry}
\usepackage{graphicx}
\usepackage{parskip}
\setlength{\parindent}{0pt}
\usepackage{minted}
\usepackage[bookmarks]{hyperref}

\begin{document}

\title{\Huge SETR 1 - Memoria de prácticas}
\author{Bermudo Garay, Pablo Manuel\\
López Moreno, José Manuel}
\date{}
\maketitle{}
\thispagestyle{empty}
\newpage

\tableofcontents{}
\newpage

\section{Práctica 1}
\subsection{Objetivos}
- Entender el esquema de la placa de Innovación Docente de ATC
- Manejar el software de desarrollo/depurador
- Manejar la utilidad de configuración de Silabs
- Entender y manejar los displays LCDs.
- Probar algunos dispositivos del microcontrolador: puertos, timer
y conversores A/D.

Un aspecto importante en el desarrollo con microcontroladores es
saber utilizar y aprovechar el software elaborado por otros y que
nos pueda venir bien al proyecto que se pretenda hacer.  Este
software elaborado previamente puede tener diferentes orígenes:
bajado de internet de la página de ejemplos del fabricante,
sintetizado por una herramienta de generación de código automática
(como la utilidad de configuración), comprado a otro desarrollador,
etc. Como es natural la documentación y claridad del código depende
mucho de su origen, de la complejidad del mismo e incluso de su
tamaño. En esta práctica vamos a utilizar código de diferentes
fuentes y adaptarlo para ir desarrollando los ejemplos que se
proponen. Por tanto va a ser más una labor de análisis y adaptación
del software existente que no el escribir muchas líneas en una
pantalla en blanco.

\subsection{Introducción}
Para el desarrollo de la práctica en primer lugar el profesor nos
pone en contexto sobre los elemento que vamos a utilizar y las
principales características del microcontrolador 8051 de Silabs el
cual es el que vamos a usar en la mayor parte de las prácticas de
la asignatura. Como elemento a destacar el 8051 viene provisto de
un ``crossbar'' el cual es el elemento que nos permite configurar
de forma muy flexible los puertos de entrada/salida del
microcontrolador, evitando así tener que usar pines predeterminados
para utilizar los recursos de los que provee el dispositivo.

Además del manejo del 8051 el profesor nos introdujo el método de
funcionamiento de los Liquid-crystal displays (LCD's).  Este
dispositivo consta de dos filas y aunque la memoria puede almacenar
hasta cuarenta columnas de caracteres en cada fila, en la pantalla
solo muestran dieciséis. Las columnas mostradas se seleccionan
mediante un puntero que apunta a la primera columna que se desea
mostrar.

\subsection{Desarrollo de la práctica}
Durante la práctica conocimos los fundamentos básicos básicos del
trabajo con microcontroladores, para ello interactuamos con los
leds y botones de la placa de innovación docente del ATC.  Además
de esto nos comunicamos con un dispositivo externo, en concreto un
display LCD al cual le enviábamos cadenas de texto usando las
funciones proporcionadas por la biblioteca destinada a tal efecto.
\inputminted[tabsize=4, fontsize=\small]{c}{prac1-main.c}

\subsection{Conclusiones}
Sí, se consiguieron los objetivos, llegando hasta el final de
la práctica, manejando los leds correspondientes y consiguiendo
interactuar con el display LCD. Como valoración personal la
práctica al ser introductoria ha sido muy poco didáctica. Esto
es debido a que el objetivo de la práctica en cuestión es
familiarizarse con los elementos de trabajo con los que vamos a
interactuar durante el resto de la asignatura, siendo la
manipulación de leds y la comunicación con el display LCD
tareas básicas dentro del marco de trabajo de la asignatura.

\section{Práctica 2}

\subsection{Objetivos}
Después de una primera práctica en la que tomamos contacto con el
mundo del desarrollo con microcontroladores, usando la placa de
innovación docente del departamento, que monta un microcontrolador
8051 de Silabs, es hora de conocer otros entornos de desarrollo. En
esta ocasión vamos a trabajar con microcontroladores Freescale
Flexis y el entorno de desarrollo CodeWarrrior.

\subsection{Introducción}
Para familiarizarnos con este entorno se pretende desarrollar una
especie de indicador de actitud que nos indique la horizontalidad
de la placa a través de leds y sonido.

Para realizar esto se dispone de una placa DEMOQE128 con un
microcontrolador Flexis 8 bits (MC9S08QE128). Esta placa tiene
además de un acelerómetro, 8 LEDs en una línea y un pequeño
altavoz.  Se pretende hacer un sistema que instalado, por eje mplo,
en un avión nos indique cuando se pierde la horizontalidad, porque
viremos a derecha o izquierda o porque subamos o bajemos (la parte
delantera de la placa sería el morro del avión, la trasera la
cola). Se trata de marcar la inclinación lateral encendiendo un
solo LED, de forma que cuando la placa se encuentre horizontal sea
unos de los LEDs centrales el encendido (o mejor los dos LEDs
centrales), y cuando se incline a la izquierda sea uno de los LEDs
de la izquierda el que se encienda, cada vez más a la izquierda,
dependiendo de la inclinación (al contrario de lo que haría un
nivel de burbuja). Por otra parte, la inclinación hacia delante y
hacia atrás (subir o bajar el avión) se marcará por un sonido, de
forma que cuando se encuentre horizontal no suene, cuando se suba
genere un sonido cada vez más grave, cuanto más se incline en
subida; y en bajada un sonido agudo, cada vez más, cuanto más
inclinada sea la bajada. Por último, si hay turbulencias
(movimientos verticales bruscos, cambios bruscos eje z del
acelerómetro) o volamos invertidos (cambio eje z) debe sonar un
doble tono intermitente. Aunque no es estrictamente necesario, se
debe utilizar un interruptor, de los que dispone la DEMOQE, para
inicializar la horizontalidad (es decir, indicarle al sistema el
``cero'').

\subsection{Desarrollo de la práctica}
Código de la práctica con la función de los LEDs realizada.
\inputminted[tabsize=4, fontsize=\small]{c}{prac2-main.c}

\subsection{Conclusiones}
Se consiguieron todos los objetivos de la práctica. Además fue una
sesión interesante ya que en ella conocimos el entorno
CodeWarrrior, más potente y a la vez más sencillo en cuanto a la
utilización del asistente de configuración que que el empleado para
el 8051 de Silabs.

\section{Práctica 3}

\subsection{Objetivos}

\begin{itemize}
    \item Conocer como conectar un microcontrolador como periférico del PC
        mediante USB, obteniendo una expansión hardware fácil en los PCs.

    \item Conocer los fundamentos de la conexión USB, en concreto términos como
        LowSpeed, FullSpeed, SuperSpeed, USB 2.0, que son los descriptores y
        los endpoints, datos con los que durante el desarrollo de las practicas
        trataremos en mas de una ocasión ya que son los aspectos mas básicos de
        los dispositivos USB.  Con los descriptores y los endpoinds nos
        permiten configurar y comunicarle al equipo el tipo de dispositivo que
        le conectamos,

    \item Realizar un dispositivo USB de clase HID para usar el inclinómetro
        como mando para PC.
\end{itemize}

\subsection{Introducción}

Para comenzar la práctica, el profesor nos recuerda los
diferentes tipos de dispositivos USB dependiendo de su uso, y
como hay que configurarlos para que trabajen de una forma u otra
para con el PC.

\subsubsection{Clasificación general de los dispositivos USB}

Los dispositivos USB se pueden clasificar de muy diversas formas, pero una
posible clasificación atendiendo a su relación con el host sería:

\begin{itemize}
    \item \textbf{Dispositivos de clase específica de vendedor:} Para éstos se
        necesita desarrollar un driver que se cargue en el sistema operativo,
        normalmente lo suministra el fabricante. Aunque también se pueden
        utilizar un driver general como el USBexpress. La principal ventaja de
        este tipo de dispositivos es que se pueden adaptar perfectamente a la
        aplicación (no tienen limitaciones de velocidad de transferencia, como
        algunos dispositivos de clase genérica, por ejemplo los HID)

    \item \textbf{Dispositivos de clase genérica:} Estos dispositivos tienen ya
        el driver en el sistema operativo. El fabricante no precisa desarrollar
        el driver, ni el usuario cargarlo. Los casos típicos son los HID,
        almacenamiento masivo, sonido, etc. A su vez podemos clasificar estos
        dispositivos de clase genérica en:
        \begin{itemize}
            \item Los que son utilizados directamente por el SO, por ejemplo un
                ratón o un teclado, no precisa ningún tipo de software
                adicional.

            \item Los que no son utilizados directamente por el SO, estos son
                definidos por el vendedor, en este caso aunque no precise
                driver, es necesario desarrollar algún tipo de aplicación para
                hacer uso del dispositivo USB.
        \end{itemize}
\end{itemize}

La placa desarrollada sirve para probar cualquiera de los tipos de dispositivos
USB comentados. Un aspecto importante para todos los dispositivos USB son los
descriptores, en particular para los dispositivos de clase genérica. Por
ejemplo, el ratón HID, que se suministra, se puede cambiar por un joystick,
sólo hay que modificar el “report descriptor” utilizando la herramienta “HID
descriptor tool”.

\subsubsection{Los dispositivos de clase HID (Human Interface Device)}

La clase HID engloba principalmente a los dispositivos que son usados por los
humanos para el control de los computadores: ratones, teclados, joysticks ...
Pero también puede englobar a otros tipos dispositivos no directamente
relacionados con estos: displays, lectores de código, termómetros, voltímetros
...  Tres son los aspectos a tener en cuenta con respecto a la clase HID:

\begin{enumerate}
    \item \textbf{Qué hace que un dispositivo sea de clase HID:}

        Como en el resto de dispositivos USB los descriptores juegan un papel
        muy importante en los de clase HID. En el Interface descriptor el campo
        bInterfaceClass debe ser 03h para indicar que el dispositivo es de la
        clase HID. Para este tipo de dispositivos sólo hay dos subclases: 1
        para dispositivos soportados por la BIOS (Boot interface subclass), 0
        para el resto. El protocolo puede ser: 1 para teclado, 2 para ratones y
        0 para ninguno.Existen además tres estructuras de descriptores
        específicos para esta clase:
        \begin{itemize}
            \item \textbf{HID descriptor:} describe el tipo y la longitud de
                los descriptores subordinados (los dos siguientes).

            \item \textbf{Report descriptor:} su presencia es obligatoria para
                los dispositivos HID, examinando este descriptor el driver del
                host puede determinar el tamaño y la composición de los datos
                mandados desde el dispositivo USB, también determina si es tipo
                joystick, ratón, etc. y por tanto como es visto por el sistema
                operativo.

            \item \textbf{Physical descriptor:} este tipo de descriptores es
                opcional en la clase HID, informa sobre que parte o partes del
                cuerpo humano son utilizados para controlar el dispositivo USB.
                En esta práctica no lo vamos a usar ni describir.
        \end{itemize}

    \item \textbf{Qué datos y cómo se transfieren en los dispositivos de
        clase HID:}

        En los dispositivos de clase HID se transfiere unas estructuras
        denominadas REPORT. Un report está compuesto por una o más
        transacciones de datos entre el dispositivo y el host, hay tres tipos:
        Input, Output y Feature (configuración). Los dispositivos HID
        obligatoriamente utilizan el Endpoint 0 de control y además un Endpoint
        tipo interrupt de entrada, opcionalmente pueden usar un Endpoint tipo
        interrupt de salida, si éste no existiese se utiliza para transmitir
        información del host al dispositivo USB el Endpoint0 de control
        contestando a la petición Set Report.

    \item \textbf{Cómo se especifican la configuración y en particular el
        formato de los datos en la clase HID:}

        Esto se hace mediante los descriptores, en particular en los
        dispositivos de clase HID el report descriptor es el encargado de
        indicar el formato de los datos (reports) que se van a transferir. Es
        importante entender que el Host lee una vez el report descriptor y no
        tiene porque volver a leerlo. A partir de ese momento cada vez que el
        Host lea (o escriba) datos se entiende que el formato es el descrito
        por dicho descriptor. El papel de los descriptores en general es sólo
        de configuración.  Para describir la estructura del report descriptor
        comenzamos mostrando el elemento mínimo que lo compone, de forma
        genérica se denomina item. Los ítems tienen un byte de prefijo que en
        el caso general contiene bits que indican el tamaño del mismo, el tipo
        y la etiqueta. En el caso del item largo el valor está definido y es
        fijo (el que se muestra).

        Los ítems son al final una serie de números de significado para el
        dispositivo e interpretables por el Host pero de muy difícil lectura
        por el desarrollador humano. Para manejar dichas claves por el
        desarrollador se asocia una especie de lenguaje de descripción, que es
        el que vamos a resumir en los siguientes párrafos.  Afortunadamente
        existe una herramienta pública (Hid descriptor tool) que permite
        “compilar” y obtener la serie numérica del report descriptor. Los Items
        lo podemos dividir en tres tipos: Main, Global y local.  Los ítems main
        a su vez son:

        \begin{itemize}
            \item \textbf{Input:} describe información que pasa del dispositivo al
                master.

            \item \textbf{Output:} describe la información que pasa del master al
                dispositivo.

            \item \textbf{Feature:} describe la información de configuración que puede
                ser enviada al dispositivo.

            \item \textbf{Collection:} agrupa varios item correspondientes a un
                dispositivo.

            \item \textbf{End Collection:} señala el fin del agrupamiento.
        \end{itemize}

        Realmente los tres primeros tipos de mains indican que hay una
        transferencia.  Si entendemos que el report es una sucesión de datos lo
        que hacen es ir sumando los datos que componen dicha sucesión. Cómo se
        componen o para qué sirven dichos datos lo describen los ítems locales
        y globales, que siempre se encuentran precediendo a los ítems mains.

        En general, en un report descriptor, los primeros item son Usage Page y
        Usage, ambos describen de forma general la funcionalidad de los campos
        siguientes.  El siguiente item es el de Collection, éste agrupa datos
        del mismo tipo contenidos desde Collection hasta End Collection. Hay
        varios tipos de Collection por ejemplo: Application, Logical y
        Physical.
\end{enumerate}

\subsubsection{Acelerómetro/inclinómetro}

El acelerómetro que vamos a utilizar es el ADXL311, este dispositivo se
encuentra soldado en la placa PCB. El ADXL311 dispone de dos salidas analógicas
que dependiendo de la aceleración proporciona una tensión que podremos medir
con el conversor a/d del microcontrolador (en la PCB esas dos salidas están
conectadas al conversor A/D del Cygnal). El funcionamiento teórico de los
acelerómetros es relativamente simple, aunque su construcción es compleja. En
este caso podríamos considerar que consiste en una masa suspendida por un
“muelle de silicio” de forma que si no está sometida a ninguna aceleración hace
que las placas de un condensador se encuentren alineadas, pero cuando hay una
aceleración (fuerza = masa x aceleración) las placas del condensador dejan de
alinearse disminuyendo la capacidad.

Esta disminución de capacidad es el parámetro eléctrico que se trata en el
interior del ADXL311 para obtener la salida en tensión en función de la
aceleración. Si la única aceleración a la que se somete este chip es la de la
gravedad puede servir para medir la inclinación, si estuviera sometido además a
otras aceleraciones el caso se complica, por tanto es conveniente mover nuestro
ratón de forma que la aceleración a que lo sometemos sea despreciable frente a
la gravedad.

\subsection{Desarrollo de la práctica}

Para el desarrollo de la practica nos centrábamos principalmente en dos
actividades, adaptar los descriptores para usar nuestra placa como ratón y/o
mando USB, indicándole entre otras cosas el tamaño de datos a enviar, el tipo
de dispositivo(ratón, o joystick según la ``altura'' a la que estuviésemos de
la realización de la practica y en el código principal el tratamiento del
acelerómetro adaptándolo a los ejes X e Y que requiere el PC como entrada de
ratón/joystick.

\inputminted[tabsize=4, fontsize=\small]{c}{prac3-desc.c}
\inputminted[tabsize=4, fontsize=\small]{c}{prac3-main.c}
\subsection{Conclusiones}

La práctica se consiguió realizarse en su totalidad llevando a cabo tanto el
ratón como el joystick, ilustrándonos en la iniciación a la creación de
alternativas USB, el manejo y funcionamiento de los descriptores, para el
manejo de dispositivo usando métodos diferentes como puede ser el acelerómetro,
dando ideas de aplicación a personas con diferentes necesidades.

\section{Práctica 4}

\subsection{Objetivos}
Esta práctica pretende ser una introducción a la técnica conocida
como modulación por ancho de pulsos, PWM por sus siglas en inglés.
Concretamente la usaremos para controlar motores de corriente
continua y servos en función a las lecturas de un acelerómetro. Con
las información de un eje del acelérometro controlaremos el motor y
con la del otro el servo. En un hipotético coche de radiocontrol
con el servo controlaríamos la dirección y con el motor la
tracción.

\subsection{Introducción}
La técnica PWM consiste en la modulación del ancho (duración) de
pulsos eléctricos y se usa principalmente para controlar la
potencia suministrada a un dispositivo eléctrico o para codificar
la información enviada a través de un canal de comunicaciones.
Tiene multitud de aplicaciones prácticas como el control de motores
eléctricos y servos o la regulación de voltaje, esta última usada
en distintos tipos de fuentes de alimentación.

Un servomotor es un dispositivo capaz de orientar su eje de salida
hacia una posición arbitraria dentro de su rango de actuación.
Estos elementos se usan principalmente en modelismo. Están
compuestos por un pequeño motor de corriente continua y un
controlador de posición integrado, al que sólo hay que
especificarle la posición (ángulo) en la que ha de situarse. Estos
dispositivos son muy fáciles de usar, ya que el controlador interno
se encarga de posicionar el servo con alta precisión, y además
contiene la circuitería necesaria para proporciona la potencia
necesaria al motor de su interior. Nosotros sólo hemos de
transferirle la posición en que queremos situar el servo, para ello
usan señales de PWM con periodos de entre 50/60Hz, codificándose la
posición del servo en el tiempo en alto de la señal PWM.

Los motores de DC son dispositivos que requieren cantidades de
potencia, muy elevadas en comparación con la potencia que es capaz
de suministrar un microcontrolador por sus puertos de
entrada/salida, así que para proporcionarles la potencia necesaria
se ha de recurrir a controladores externos, como son los puentes en
H. Los puentes en H se suelen componer de 4 transistores de
potencia, capaz de proporcionar varios amperios (4A) a altos
voltajes (12V). La idea es que estos transistores sean conmutados
desde el microcontrolador, haciendo que la intensidad fluya entre
los bornes del motor, además, gracias a su topología, el flujo de
intensidad se puede invertir, pudiendo hacer rotar el motor en
ambos sentidos. La velocidad de los motores de DC es proporcional
al voltaje aplicado en sus bornes.

Para obtener velocidades intermedias se suelen usar señales de PWM
con frecuencias altas para conmutar los puentes en H. De manera que
el puente no está conmutado constantemente, si no que se usa una
señal de PWM para ``encenderlo'' y ``apagarlo'' muy rápidamente,
pareciendo desde el punto de vista del motor que la tensión
aplicada a sus bornes es proporcional duty-cycle de la señal de
PWM, gracias a que el motor se comporta como un filtro paso de
baja.

\subsection{Desarrollo de la práctica}
Durante la práctica se desarrollo un pequeño programa para el
microcontrolador 8051 que lee dos ejes del acelerómetro y usa estos
valores para controlar la posición del servo y la velocidad del
motor de corriente continua.
\inputminted[tabsize=4, fontsize=\small]{c}{prac4-main.c}

\subsection{Conclusiones}
Fue una práctica muy ilustrativa sobre los usos prácticos de
modulación por anchura de pulso. Se consiguió completarla entera,
no sin un poco de desconcierto por culpa del material de
laboratorio (el motor DC concretamente) que estaba defectuoso y
presentaba un comportamiento errático.


\section{Práctica 5}

\subsection{Objetivos}
La práctica 5 pretende introducirnos en el campo de las
comunicaciones inalámbricas mediante modems XBEE, enviando y
recibiendo mensajes entre nodos y mostrándolos posteriormente en un
display LCD. Para ello conoceremos el método de comunicación con el
que trabajan estos modems.

Como objetivo práctico se implementa un sistema de localización de
interiores basado en la potencia en la que reciben los datos los
diferentes nodos situados en el aula.

\subsection{Introducción}
Para la práctica 5 disponemos de 3 nodos fijos con los que nos
podemos comunicar, dos balizas, situadas en esquinas contrarias del
laboratorio que nos permitirán en la segunda parte de la práctica
conocer la localización de nuestro puesto y un nodo situado en el
PC del profesor, el cual dispone de una aplicación que mostrará los
mensajes enviados por los dispositivos de los alumnos a ese
terminal.

El modem XBEE se comunica a través del puerto serie asíncrono y
tiene dos modos de funcionamiento, aunque el modo API no lo usamos.

El modo transparente consiste en enviar la información únicamente
entre nodo emisor y nodo destino. Este modo, nos permite pasar al
modo AT, el cual nos da la posibilidad de comunicarnos con nuestro
propio dispositivo, para así poder darle al dispositivo las
instrucciones correspondientes de ejecución que necesita recibir.
Este cambio de modo se hace mediante el envío de tres ``+''
consecutivos (+++).

En nuestro caso, para comunicarnos con las dos balizas, debemos
entrar en modo AT, indicarle al XBEE que nos vamos a comunicar con
la primera baliza (con atdl), salir del modo AT, realizar la
comunicación con la primera baliza, y realizar la misma tarea para
la baliza siguiente.

Durante nuestra práctica, desarrollada durante dos sesiones, se
divide en dos partes, la primera realizando al comunicación con el
PC del profesor y enviarle un mensaje, y la segunda, mediante la
comunicación con las balizas comentadas anteriormente, y con una
tabla previamente elaborada, conocer en que posición del
laboratorio nos encontramos, comparando la potencia de la señal
recibida por cada una de las balizas con la tabla de valores ya
muestreada permitiendo conocer dependiendo de la potencia que da
cada una, la distancia a ellas y a partir de ellos, la posición.

\subsection{Desarrollo de la práctica}
Para el desarrollo de la práctica en primer lugar configuramos el
microcontrolador para usarlo con 9600 baudios, que es el bit rate
en el cual trabajan los modems.

Posteriormente creamos el código para comunicarnos con la baliza
del profesor y el mensaje que enviaremos. De forma análoga se
prepara la recepción del mensaje de confirmación que nos indica
el PC del profesor y mostrarlos por el LCD.  En la segunda parte
de la práctica tratamos mediante cambios entre modo AT y modo
transparente la alternancia en la comunicación con una baliza y
otra, y   el envío y recepción de mensajes.

\inputminted[tabsize=3, fontsize=\small]{c}{prac5-config.c}
\inputminted[tabsize=4, fontsize=\small]{c}{prac5-main1.c}
\inputminted[tabsize=4, fontsize=\small]{c}{prac5-main2.c}

\subsection{Conclusiones}
La práctica se consiguió hacer en su totalidad y pese a una
lectura del puesto en el que estábamos algo difusa, consiguió su
cometido dándonos una introducción muy interesante del trabajo
con comunicaciones inalámbricas y estos modems XBEE, el trabajo
de comunicación interna con el propio dispositivo y con
dispositivos externos y nos da una ligera idea de las diferentes
aplicaciones que se le pueden dar en sistemas de comunicación con
microcontroladores sin la necesidad de cableado.

\section{Práctica 6}

\subsection{Objetivos}
Esta práctica pretende introducir el trabajo con dos tipos de
puertos serie desde el microcontrolador.  Por un lado se va a usar
el clásico puerto serie RS232, en la actualidad casi en desuso.
Además se explicará como hacer uso de los puerto USB de clase
específica de fabricante. Tambien realizaremos un ejercicio híbrido
ya que usaremos puertos COM virtuales que emulan el viejo RS232
sobre una interfaz física USB. Para ello usaremos un firmware USB
de clase HID.

\subsection{Introducción}
A la hora de acceder a los puertos y dispositivos del PC bajo
windows se pueden utilizar diferentes mecanismos. No es objetivo de
esta práctica explicar como se desarrollan drivers bajo windows,
sólo se va a mostrar aquellos aspectos básicos que nos ayuden a
realizar la práctica.  El sistema operativo se puede considerar
dividido en múltiples niveles, sin embargo, para nuestros objetivos
basta considerar sólo dos: kernel y usuario. Sólo el núcleo del
sistema (kernel) puede realizar las operaciones de entrada/salida,
en él se debe localizar el manejador (driver) que permita a las
aplicaciones utilizar los dispositivos o puertos. Las aplicación es
se pueden comunicar con los drivers (modo Kernel) de diferentes
maneras, pero hay una que es básica y se utiliza prácticamente en
todos los sistemas operativos (incluido UNIX, norma ANSI), es
mediante las funciones de la API para el acceso a ficheros:
CreateFile, ReadFile y WriteFile. En esta práctica se va a utilizar
dicho mecanismo. En el caso del puerto serie COM real el driver
está ya en el kernel del windows, basta utilizar estas funciones
como se muestra en el apéndice A. Sin embargo, el puerto USB no
implementa dicho driver de forma completa, sí hay un USB, pero
falta una parte en modo kernel que hay que desarrollar e instalar:
el driver de clase. Para facilitar dicha labor de desarrollo en el
caso del USB se va a utilizar un ``wizard'' que automáticamente nos
va a generar el esqueleto del driver de clase, que completaremos
con unas pocas líneas.  En la explicación previa a la práctica se
detallará cómo se prepara el driver, ahora es importante darse
cuenta que nuestra aplicación para el PC puede ser casi la misma
para todos los casos, ya que se utilizan las mismas funciones para
conectarse con el driver.  Tanto para rs232 como para USB hay
diferentes alternativas a las expuestas. Por ejemplo en el caso de
rs232 se podría hacer bajo msdos y acceder directamente al puerto.
En el caso del USB se puede diseñar el sistema microcontrolador
como un HID (dispositivo de interface humana), como se ve en otras
prácticas de la asignatura (esto se recordará al principio de la
práctica).  La estructura, desde el punto de vista del sistema
operativo, del software de manejo de los puertos USB es la que se
muestra en la siguiente figura. El USB DRIVER ya está en el sistema
operativo, lo que se tiene que construir en nuestro caso es Class
Device Driver, si fuese de clase genérica, como la HID, no se
necesitaría construir dicho driver.

Como hemos indicado antes el driver de clase será ``interpelado''
desde la aplicación mediante las llamadas a la API de manejo de
ficheros (CreateFile, ReadFile, etc.). Una forma simple de ver un
driver es como un conjunto de subrutinas o funciones (parecido a
una dll), cada vez que la aplicación ``llama'' al driver realmente
se lanza una (o varias) subrutina de éste. Lo que realmente ocurre
es que se construye una estructura denominada IRP (I/O Request
Packet) que se le pasa a la función correspondiente del driver,
observad la siguiente figura. El driver de clase que se va a
construir tendrá que ``llamar'' al USB Driver y compartir de alguna
forma los datos con él , esto se hace mediante una especie de IRP
denominada URB (USB Request Block). En la práctica veremos que lo
que tenemos que hacer es simplemente ``construir'' el URB con la
información necesaria y ``presentarlo o entregarlo'' al USB driver.

\subsection{Desarrollo de la práctica}
A continuación se muestra el archivo con el código desarrollado
para el apartado uno, la conexión sobre RS232.
\inputminted[tabsize=4, fontsize=\small]{c}{prac6-main.c}

Y a continuación el archivo con los descriptores USB sobre el que
se trabajaba en los apartados dos y tres.
\inputminted[tabsize=4, fontsize=\small]{c}{prac6-desc.c}

\subsection{Conclusiones}
Aunque un poco apurado debido a su longitud se consiguieron
realizar los distintos apartados de la práctica con éxito.

Es una práctica muy interesante ya que se explica un tema muy
importante en el mundo de los microcontroladores como es el uso de
puertos para comunicarse con otros dispositivos.  Pero precisamente
lo ``densa'' que es la práctica hace la mayoría de los conceptos
queden un poco cogidos con pinzas a pesar de haberse comprendido.

\section{Práctica 7}

\subsection{Objetivos}
El objetivo de esta practica es la introducción y el trabajo con
los sistemas empotrados en tiempo real y el gobierno de este
mediante un sistema operativo Freertos.  Como medio de desarrollo
para conseguir esta inicialización dispondremos de un sistema
basado en ARM7 sobre una placa de desarrollo Makingthings,
dándonos las ventajas de las que nos proporciona esa placa de
desarrollo propias de ellas mismas que nos permite manejar de
forma simple el sistema operativo en tiempo real Freertos.

Nos introduciremos también en el entorno de desarrollo para este
tipo de dispositivos Crossworks, el cual mediante el sistema JLINK
nos permite depurar el código introducido en la placa.  Como
objetivo de la práctica intentaremos conseguir comunicarnos con un
display LCD y el manejo de un servo, mediante la realización de
tareas.

\subsection{Introducción}
El sistema operativo Freertos nos permite mediante una serie de
bibliotecas, y su core, interactuar en tiempo real con
dispositivos y salidas de la placa de desarrollo, usando en
nuestro caso la placa Makingthings, y ARM7 de Atmel.  Este
sistema operativo nos ofrece diferentes paquetes dependiendo de
las funcionalidades que queramos obtener de base, siendo estos
más o menos pesados según nuestra elección.  Para nuestras
practicas utilizaremos el paquete Heavy para no preocuparnos en
las limitaciones ya que este paquete es el más completo del SO.
Como primer ejercicio desarrollaremos mediante el uso de tareas
el manejo de un display LCD, creando tareas que ejecuten dos
mensajes diferentes usando la misma función, en una linea
diferente del display. Al hacerlo de directa, como no se gestiona
el acceso a los comandos y caracteres del display, este imprime
los datos de forma errónea, solapándose unos con otros, ya que no
hay ningún medio que gestione que una tarea haya acabado antes de
cambiar de fila de escritura, para ellos, debemos utilizar
semáforos para proteger esa escritura, habilitando así el
correcto orden de escritura.  Para el segundo ejercicio,
utilizaremos el manejo de un servo para entrar en detalle un poco
más del uso de las tareas, ya que con las tareas compartidas en
tiempo real, podremos usar ese servo y dos potenciómetros para,
mediante la medida de ellos, en tiempo real establecer la
velocidad a la que gira el servo y con el otro potenciómetro la
dirección de giro.  Cabe decir, que en todo el trabajo de Tiempo
real, siempre esta bien tener una tarea con la mínima prioridad
que controle un parpadeo constante de uno de los LEDs de la
placa, para mediante el cual, poder controlar el correcto
funcionamiento de las tareas, ya que si esta tarea de menor nivel
funciona correctamente, significa que las de mayor nivel también
tienen tiempo para funcionar de forma apropiada.

\subsection{Desarrollo de la práctica}
La práctica se desarrolla en primer lugar configurando las tareas
del LCD, añadiéndole posteriormente el tratamiento con semáforos
para controlar la correcta ejecución y tratamiento del recurso
compartido, y siempre tener en cuenta la función blinktask, que
nos permite usar el LED como depurador ``visual''. Para esto
usamos las bibliotecas que nos proporciona el SO sobre semáforos,
pudiéndose localizar fácilmente su funcionamiento en la
documentación facilitada por el SO.
\inputminted[tabsize=4, fontsize=\small]{c}{prac7-makedisplay.c}

Para el caso de la segunda parte, usando la entrada analógica
para medir los potenciómetros, y actualizar el comportamiento del
servo según la respuesta obtenida de esos potenciómetros,
conseguimos nuestro cometido, sin olvidarnos nunca de seguir
usando la función blinktask.
\inputminted[tabsize=4, fontsize=\small]{c}{prac7-makeservo.c}

\subsection{Conclusiones}
Como conclusión la práctica 7 fue muy interesante ya que nos
adentraba en el mundo de los sistemas empotrados en tiempo real
gestionados mediante sistemas operativos destinado para ello, lo
cual no habíamos visto hasta la fecha en la titulación, y que nos
deja un gran sabor de boca y con ganas de profundizar más en la
cuestión en la continuación de esta asignatura.  En la práctica
se consiguió acabar en su totalidad y nos permitió conocer como
mediante la gestión de tareas, podemos gestionar una serie de
tratamiento de funciones o procedimientos con una serie de
prioridades asignadas que nosotros creamos convenientes para
nuestros proyectos.

\section{Práctica 8}

\subsection{Objetivos}
El objetivo de esta práctica es servir como presentación del
funcionamiento de los dispositivos USB de clase audio y la
generación de sonido mediante timers PWM. Se usará la placa
``makingthings'' para realizar un dispositivo capaz de generar
comandos MIDI que que se reproduzcan en el PC y otro que reciba
comandos MIDI y genere el sonido mediante un altavoz haciendo uso
de señales PWM.

\subsection{Introducción}
Una de las clases de dispositivo más interesantes del estándar USB
es la clase audio. El estándar define 3 subclases:
\begin{itemize}
    \item \textbf{AudioControl:} Permite comunicarse con diversos
        controles de audio.
    \item \textbf{AudioStreaming:} Para la transferencia de sonido
        en sí.
    \item \textbf{MIDIStreaming:} Sirve para transferir comandos
        MIDI.
\end{itemize}
También proporciona un valor para identificar subclases no definidas en el
estándar.

El estándar MIDI (Musical Instrument Digital Interface) es un protocolo que
permite a distintos tipos de dispositivos electrónicos musicales comunicarse
entre sí usando una serie de comandos que definen diversos aspectos del sonido.

Durante la práctica el PC y su “piano virtual” conectarán con la placa de
Makingthings mediante USB, esto implica que el PC mandará comandos Midi para
controlar la placa y configurarla de forma adecuada, esta información mandada
desde el PC si no se extrae del buffer USB de la placa hará que su firmware USB
acabe mandando constantemente NACK al PC, si es ta situación se prolonga
durante un tiempo el PC dará por no válido nuestro sistema. Para evitar este
problema, tanto para durante los dos apartados de la práctica, hay que leer
mediante USB\_Read lo que el PC mande, no es necesario interpretar todo lo que
manda el PC, sólo tenemos que vaciar el buffer para que no se llegue a la
situación descrita, en caso contrario el sistema se podría bloquear. Por
ejemplo, se trata de que si llega un comando desde el PC de cambio de
programa/instrumento, nuestra placa lo saca del buffer (lectura USB) y no hace
nada, el PC dará por válido que se ha cambiado de instrumento, aunque en el
apartado dos nuestro sistema siga sonando de la misma forma y no con el
instrumento nuevo que pide el PC.

\subsection{Desarrollo de la práctica}
Se realizó un programa para el microcontrolador de la placa que
generaba comandos MIDI con instrumentos distintos en dos
situaciones, al apretar un pulsador y al detectar ciertos
movimiento con el acelerómetro.
\inputminted[tabsize=4, fontsize=\small]{c}{prac8-main.c}

\subsection{Conclusiones}
Finalmente por motivos de tiempo, el segundo apartado de la
práctica, consistente en reproducir sonidos modulados mediante PWM
en un pequeño altavoz conectado a la placa, no se llego a realizar
ya que ni siquiera llegó a ser explicada. Por lo demás consideramos
muy interesante la sesión, aprendimos mucho sobre los dispositivos
USB de clase audio y sus distintas aplicaciones, así como sobre el
estándar MIDI.

\end{document}

